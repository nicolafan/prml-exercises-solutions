\section{Introduction}

\paragraph{Exercise 1.1}

We seek to minimize $E$ by setting its derivative with respect to $\textbf{w}$
to zero. We can consider each weight $w_i$ separately, so we have

\begin{align*}
    \frac{\partial E}{\partial w_i} = 0 &\iff
    \sum_{n=1}^N \{[ w_0 + w_1x_n + w_2x_n^2 + \dots + w_Mx_n^M - t_n]x_n^i\} = 0\\
    &\iff \sum_{n=1}^N \{w_0x_n^i + w_1x_n^{i+1} + w_2x_n^{i+2} + \dots + w_Mx_n^{i+M}\} = \sum_{n=1}^{N} (x_n)^it_n\\
    &\iff \sum_{j=0}^M \{\sum_{n=1}^N (x_n)^{i + j} w_j\} = \sum_{n=1}^N (x_n)^it_n
\end{align*}

This result corresponds to the set of linear equations shown in the text.
    
\paragraph{Exercise 1.2}

Again, we seek to minimize $\tilde{E}$, by setting each one of the partial derivatives
with respect to $w_i$ to zero. We have

\begin{align*}
    \frac{\partial\tilde{E}}{\partial w_i} = 0 &\iff \sum_{n=1}^N \{[y(x_n, \textbf{w}) - t_n]x_n^i\} + \lambda w_i = 0\\
    &\iff \sum_{n=1}^N \{[w_0 + w_1x_n + w_2x_n^2 + \dots + w_Mx_n^M - t_n]x_n^i\} + \lambda w_i = 0\\
    &\iff \sum_{n=1}^N \{w_0x_n^i + w_1x_n^{i+1} + w_2x_n^{i+2} + \dots + w_Mx_n^{i+M}\} + \lambda w_i = \sum_{n=1}^N (x_n)^it_n\\
\end{align*}

The coefficients $\textbf{w} = \{w_i\}$ that minimize $\tilde{E}(\textbf{w})$ are given by the solution
to the following set of linear equations:

\begin{equation*}
    \sum_{j=0}^M A_{ij}w_j + \lambda w_i = T_i
\end{equation*}

where $A_{ij} = \sum_{n=1}^N (x_n)^{i+j}$ and $T_i = \sum_{n=1}^N (x_n)^it_n$.


\paragraph{Exercise 1.3}

We want to determine the probability $p(apple)$ of randomly extracting an apple from
a randomly chosen box.

By the law of total probability, we have

\begin{equation*}
    p(apple) = p(apple, r) + p(apple, b) + p(apple, g)
\end{equation*}

where $r$, $b$ and $g$ denote the red, blue and green boxes, respectively.

We also apply the product rule to each term of the sum, obtaining:

\begin{align*}
    p(apple) &= p(apple|r)p(r) + p(apple|b)p(b) + p(apple|g)p(g)\\
    &= \frac{3}{10}\frac{1}{5} + \frac{1}{2}\frac{1}{5} + \frac{3}{10}\frac{3}{5}\\
    &= \frac{3}{50} + \frac{1}{10} + \frac{9}{50}\\
    &= \frac{17}{50}
\end{align*}

Furthermore, the exercise asks use to determine the probability of having selected the]
green box, given that we have extracted an orange, corresponding to $p(g|orange)$.

We can use Bayes' theorem to obtain

\begin{equation*}
    p(g|orange) = \frac{p(orange|g)p(g)}{p(orange)}
\end{equation*}

We already know $p(g)$, and we can compute $p(orange)$ in the same way we computed $p(apple)$,
obtaining

\begin{align*}
    p(orange) &= p(orange, r) + p(orange, b) + p(orange, g)\\
    &= p(orange|r)p(r) + p(orange|b)p(b) + p(orange|g)p(g)\\
    &= \frac{4}{10}\frac{1}{5} + \frac{1}{2}\frac{1}{5} + \frac{3}{10}\frac{3}{5}\\
    &= \frac{2}{25} + \frac{1}{10} + \frac{9}{50}\\
    &= \frac{18}{50} = \frac{9}{25}
\end{align*}

Finally, we obtain:

\begin{align*}
    p(g|orange) &= \frac{p(orange|g)p(g)}{p(orange)}\\
    &= \frac{\frac{3}{10}\frac{3}{5}}{\frac{9}{25}}\\
    &= \frac{3}{10}\frac{3}{5}\frac{25}{9}\\
    &= \frac{225}{450} = \frac{1}{2}
\end{align*}

\paragraph{Exercise 1.4}